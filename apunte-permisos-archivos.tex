%%% LaTeX Template: Article/Thesis/etc. with colored headings and special fonts
%%%
%%% Source: http://www.howtotex.com/

\documentclass[12pt]{article}


\usepackage{apuntes-estilo}
\usepackage{fancyhdr,lastpage}
\usepackage{color,colortbl}



\def\maketitle{

% Titulo 
 \makeatletter
 {\color{bl} \centering \huge \sc \textbf{
Archivos, tipos y atributos \\
\large \vspace*{-8pt} \color{black} Una introducción a las atributos de archivos en GNU/Linux
 \vspace*{8pt} }\par}
 \makeatother


% Autor
 \makeatletter
 {\centering \small 
 	Departamento de Ingeniería de Computadoras \\
 	Facultad de Informática - Universidad Nacional del Comahue \\
 	\vspace{20pt} }
 \makeatother

}

% Custom headers and footers
\fancyhf{} % clear all header and footer fields
\fancypagestyle{plain}{\fancyhf{}}
  	\pagestyle{fancy}
 	\lhead{\footnotesize Archivos, tipos y atributos - Departamento de Ingeniería de Computadoras}
 	\rhead{\footnotesize \thepage\ }	% "Page 1 of 2"

\def\ti#1#2{\texttt{#1} & #2 \\ }



\begin{document}

\thispagestyle{empty}
\maketitle
\setlength{\parindent}{0pt}


\section*{Introducción}
Este documento describe los conceptos básicos sobre propiedades de archivos, 
incluyendo tipos y permisos de archivos, propietarios y grupos. 

\section*{Tipos de archivos}

Un archivo es una secuencia de bytes (un byte es un pequeño trozo de 
información, normalmente compuesto por ocho bits. Para nuestros propósitos
durante este texto, un byte es equivalente a un caracter). El sistema no impone 
estructura alguna a los archivos, ni asigna significado a su contenido; el 
significado de los bytes depende únicamente de los programas que 
interpretan el archivo. 

Vimos en apuntes anteriores que los archivos se organizan dentro de sistemas de 
archivos (ext3, ext4,NTFS, etc). Estos sistemas de archivos son el modo de 
ordenar los archivos dentro de los distintos medios de almacenamiento (discos
rígidos, de estado sólido, memorias flash, etc). 

En general, la mayoría de los sistemas de archivos de sistemas de tipo UNIX
definen seis tipos de archivos:

\begin{itemize}
\item Archivos regulares (Ej. un archivo de texto, un pdf son archivos regulares)
\item Directorios
\item Links simbólicos
\item Dispositivos (de bloque y caracter) 
\item UNIX Sockets  
\item Pipes 
\end{itemize}

Limitaremos nuestro estudio a los tres primeros tipos de archivos. 


\subsection*{Archivos regulares}
No son otra cosa que una bolsa de bytes: UNIX no impone ninguna estructura a
su contenido. Los archivos de texto, de datos, programas ejecutables, bibliotecas
compartidas, son todos archivos regulares.

La creación de archivos regulares sucede a través de distintos programas de 
aplicación, por ejemplo cuando utilizamos un software de ofimática, cuando 
descargamos archivos del email, a través de un editor de texto, etc. El formato 
interno de archivos regulares dependerá entonces de la aplicación que lo creó. 

La eliminación de archivos regulares sucede a través del comando \texttt{rm}. 

\subsection*{Directorios}
Los directorios son archivos que contienen referencias a otros archivos. 
Podemos crear directorios a través del comando \texttt{mkdir}, y eliminarlos
a través del comando \texttt{rmdir} si está vacío, o \texttt{rm -r} si tiene 
contenido. 

La estructura de un archivo de tipo directorio dependerá del sistema de 
archivos que usemos en particular. En general las entradas ``.'' y ``..''
se refieren al directorio actual o su padre respectivamente.  

\fcolorbox{black}{grey}{
\parbox[t]{1.0\linewidth}{ \vspace*{0.4cm}
{\bf Ejemplo} \\
\texttt{ 
\$ mkdir dir1 \\
\$ cd dir1\/ \\
\$ pwd \\
/home/lechnerm/test/dir1 \\
\$ touch arch\_dir1.txt  \\
\$ ls -l . \\
total 0 \\
-rw-r--r-- 1 lechnerm lechnerm 0 abr  4 20:06 arch\_dir1.txt \\
\$ mkdir dir2  \\
\$ cd dir2/ \\
\$ pwd \\
/home/lechnerm/test/dir1/dir2 \\
\$ touch arch\_dir2.txt \\
\$ ls -la . \\
total 8 \\
drwxr-xr-x 2 lechnerm lechnerm 4096 abr  4 20:07 . \\
drwxr-xr-x 3 lechnerm lechnerm 4096 abr  4 20:07 .. \\
-rw-r--r-- 1 lechnerm lechnerm    0 abr  4 20:07 arch\_dir2.txt \\
\$ ls -la .. \\
total 12 \\
drwxr-xr-x 3 lechnerm lechnerm 4096 abr  4 20:07 . \\
drwxr-xr-x 3 lechnerm lechnerm 4096 abr  4 20:06 .. \\
-rw-r--r-- 1 lechnerm lechnerm    0 abr  4 20:06 arch\_dir1.txt \\
drwxr-xr-x 2 lechnerm lechnerm 4096 abr  4 20:07 dir2 \\
\$ 
}

Nota: el comando \texttt{touch} crea un archivo regular vacío (o bytes). 
\vspace*{0.4cm} } }


Los \textit{nombres de archivos} contenidos dentro de un directorio, se encuentran 
almacenados en el directorio y no en el archivo en sí mismo. Esto permite, entre otras
cosas, que un mismo archivo tenga varios nombres distintos \textit{dentro del mismo 
sistema de archivos}. Cada nombre es sólo una referencia, el contenido del archivo 
será el mismo cualquiera sea el nombre que usemos para accederlo. Esto es conocido
como ``hard link''. No debe confundirse un \textit{hard link} con una copia de archivo
por ejemplo creada a través del comando \texttt{cp}. 

Un \textit{hard link} se crea a  través del comando \texttt{ln} y se elimina igual 
que cualquier otro archivo regular, a través de el comando \texttt{rm}. 

\fcolorbox{black}{grey}{
\parbox[t]{1.0\linewidth}{ \vspace*{0.4cm}
{\bf Ejemplo} \\
\texttt{
\$ ls lunes.txt  \\
ls: no se puede acceder a lunes.txt: No existe el fichero o el directorio \\
\$ echo Hoy es lunes  \textgreater  lunes.txt  \\
\$ cat lunes.txt  \\
Hoy es lunes \\
\$ ls martes.txt \\
ls: no se puede acceder a martes.txt: No existe el fichero o el directorio \\
\$ ln lunes.txt martes.txt \\
\$ cat martes.txt  \\
Hoy es lunes \\
\$ echo Hoy es martes \textgreater \textgreater  martes.txt  \\
\$ cat lunes.txt  \\
Hoy es lunes \\
Hoy es martes \\
\$ rm lunes.txt  \\
\$ ls lunes.txt \\
ls: no se puede acceder a lunes.txt: No existe el fichero o el directorio \\
\$ cat martes.txt  \\
Hoy es lunes \\
Hoy es martes \\
\$  
}
\vspace*{0.4cm} } }

El sistema de archivos mantendrá la cuenta de los ``hard links'' que existen 
a un mismo archivo, decrementando el valor cada vez que se elimina una de las 
referencias, por ejemplo a través del comando \texttt{rm}. Sólo se liberará el espacio 
utilizado por el archivo cuando éste sea la última referencia. Podemos ver el 
número de referencias a un archivo con el comando \texttt{stat} o el segundo campo 
en la salida del comando \texttt{ls -l}.  

\fcolorbox{black}{grey}{
\parbox[t]{1.0\linewidth}{ \vspace*{0.4cm}
{\bf Ejemplo} \\
\texttt{
\$ cat martes.txt  \\
Hoy es lunes \\
Hoy es martes \\
}
\vspace*{0.4cm} } }

\fcolorbox{black}{grey}{
\parbox[t]{1.0\linewidth}{ \vspace*{0.4cm}
\texttt{
\$ stat martes.txt  \\
  Fichero: «martes.txt» \\
  Tamaño: 27        	Bloques: 8          Bloque E/S: 4096   fichero regular \\
Dispositivo: fe01h/65025d	Nodo-i: 5638106    \textbf{Enlaces: 1} \\
Acceso: (0644/-rw-r--r--)  Uid: ( 1000/lechnerm)   Gid: ( 1000/lechnerm) \\
      Acceso: 2014-04-05 15:22:18.812061851 -0300 \\
Modificación: 2014-04-05 15:21:12.881692767 -0300 \\
      Cambio: 2014-04-05 15:22:13.716187888 -0300 \\
    Creación: - \\
\$ ln martes.txt lunes.txt  \\
\$ stat martes.txt  \\
  Fichero: «martes.txt» \\
  Tamaño: 27        	Bloques: 8          Bloque E/S: 4096   fichero regular \\
Dispositivo: fe01h/65025d	Nodo-i: 5638106    \textbf{Enlaces: 2} \\
Acceso: (0644/-rw-r--r--)  Uid: ( 1000/lechnerm)   Gid: ( 1000/lechnerm) \\
      Acceso: 2014-04-05 16:03:39.991411489 -0300 \\
Modificación: 2014-04-05 15:21:12.881692767 -0300 \\
      Cambio: 2014-04-05 16:03:36.739495849 -0300 \\
    Creación: - \\
\$ stat lunes.txt  \\
  Fichero: «lunes.txt» \\
  Tamaño: 27        	Bloques: 8          Bloque E/S: 4096   fichero regular \\
Dispositivo: fe01h/65025d	Nodo-i: 5638106     \textbf{Enlaces: 2} \\
Acceso: (0644/-rw-r--r--)  Uid: ( 1000/lechnerm)   Gid: ( 1000/lechnerm) \\
      Acceso: 2014-04-05 16:03:39.991411489 -0300 \\
Modificación: 2014-04-05 15:21:12.881692767 -0300 \\
      Cambio: 2014-04-05 16:03:36.739495849 -0300 \\
    Creación: - \\
\$ rm martes.txt  \\
\$ stat lunes.txt  \\
  Fichero: «lunes.txt» \\
  Tamaño: 27        	Bloques: 8          Bloque E/S: 4096   fichero regular \\
Dispositivo: fe01h/65025d	Nodo-i: 5638106     \textbf{Enlaces: 1} \\
Acceso: (0644/-rw-r--r--)  Uid: ( 1000/lechnerm)   Gid: ( 1000/lechnerm) \\
      Acceso: 2014-04-05 16:03:39.991411489 -0300 \\
Modificación: 2014-04-05 15:21:12.881692767 -0300 \\
      Cambio: 2014-04-05 16:04:09.702640844 -0300 \\
    Creación: - \\
\$ 
}
\vspace*{0.4cm} } }

Es importante entender que los ``hard link'' no son un tipo de archivos especial. 
Simplemente el sistema de archivos permite crear múltiples referencias al 
mismo conjunto de bytes. Tanto el contenido como todos las propiedades del archivo
(propiedad, permisos, tiempos de acceso, etc). son compartidos entre todos los 
``hard link''.  

\subsection*{Links simbólicos}
Un \textit{link simbólico} o ``soft link'' apunta a un \textit{nombre de archivo}.
Es decir el contenido de un archivo de tipo link simbólico, es una ruta 
(en inglés ``path'') a otro archivo. La diferencia entre un ``hard link'' y un 
``soft link'' es que, el ``hard link'' es un referencia directa a un archivo, 
mientras que un ``soft link'' es una referencia por nombre; \textit{el link simbólico
es un archivo diferente del archivo al que apunta}.  


Los links simbólicos se crean con el comando \texttt{ln -s} (se agrega la opción -s
-de soft- al comando utilizado para crear hard links), y se eliminan con el comando 
\texttt{rm}. La creación o eliminación de un link simbólico no altera la cuenta de 
enlaces del archivo referenciado. 

\fcolorbox{black}{grey}{
\parbox[t]{1.0\linewidth}{ \vspace*{0.4cm}
{\bf Ejemplo} \\
\texttt{ 
\$ ls jueves.txt \\
ls: no se puede acceder a jueves.txt: No existe el fichero o el directorio \\
\$ echo Hoy es jueves \textgreater jueves.txt  \\
\$ cat jueves.txt  \\
Hoy es jueves \\
\$ stat jueves.txt  \\
  Fichero: «jueves.txt» \\
  Tamaño: 14        	Bloques: 8          Bloque E/S: 4096   fichero regular \\
Dispositivo: fe01h/65025d	Nodo-i: 5638480     \textbf{Enlaces: 1} \\
Acceso: (0644/-rw-r--r--)  Uid: ( 1000/lechnerm)   Gid: ( 1000/lechnerm) \\
      Acceso: 2014-04-05 18:55:07.134982820 -0300 \\
Modificación: 2014-04-05 18:55:01.679118415 -0300 \\
      Cambio: 2014-04-05 18:55:01.679118415 -0300 \\
    Creación: - \\
\$ ln -s jueves.txt viernes.txt \\
\$ ls -l * \\
-rw-r--r-- 1 lechnerm lechnerm 14 abr  5 18:55 jueves.txt \\
\textbf{lrwxrwxrwx 1 lechnerm lechnerm 10 abr  5 18:55 viernes.txt -\textgreater jueves.txt} \\
\$ cat viernes.txt  \\
Hoy es jueves \\
}
\vspace*{0.4cm} } }

\fcolorbox{black}{grey}{
\parbox[t]{1.0\linewidth}{ \vspace*{0.4cm}
{\bf Ejemplo} \\
\texttt{ 
\$ stat jueves.txt viernes.txt  \\
  Fichero: «jueves.txt» \\
  Tamaño: 14        	Bloques: 8          Bloque E/S: 4096   fichero regular \\
Dispositivo: fe01h/65025d	Nodo-i: 5638480     \textbf{Enlaces: 1} \\
Acceso: (0644/-rw-r--r--)  Uid: ( 1000/lechnerm)   Gid: ( 1000/lechnerm) \\
      Acceso: 2014-04-05 18:55:07.134982820 -0300 \\
Modificación: 2014-04-05 18:55:01.679118415 -0300 \\
      Cambio: 2014-04-05 18:55:01.679118415 -0300 \\
    Creación: - \\
  Fichero: «viernes.txt» -> «jueves.txt» \\
  Tamaño: 10        	Bloques: 0          Bloque E/S: 4096   enlace simbólico \\
Dispositivo: fe01h/65025d	Nodo-i: 5638483     \textbf{Enlaces: 1} \\
Acceso: (0777/lrwxrwxrwx)  Uid: ( 1000/lechnerm)   Gid: ( 1000/lechnerm) \\
      Acceso: 2014-04-05 18:55:40.738147781 -0300 \\
Modificación: 2014-04-05 18:55:34.710297555 -0300 \\
      Cambio: 2014-04-05 18:55:34.710297555 -0300 \\
    Creación: - \\
\$ echo Hoy es viernes \textgreater \textgreater viernes.txt  \\
\$ cat jueves.txt  \\
Hoy es jueves \\
Hoy es viernes \\
\$ rm viernes.txt  \\
\$  
}
\vspace*{0.4cm} } }

Al almacenar nombres de archivos los links simbólicos, a diferencia de los ``hard 
links'', \textit{pueden hacer referencia a archivos en un sistema de archivos diferente}. A su 
vez, estas rutas de archivos pueden ser absolutas o relativas\footnote{Una ruta absoluta
es aquella que comienza a partir del directorio raíz /, ej. /usr/share/. Una ruta relativa
no contiene el directorio raíz / y es relativa al directorio de trabajo actual, ej. share/}. 

Es un error común pensar que el primer argumento del comando \texttt{ln -s} tiene algo que 
ver con el directorio de trabajo actual. Este no es el caso, \texttt{ln} simplemente tomará 
ese argumento literalmente como el destino del link simbólico. 

Se debe tener en cuenta que al eliminar el archivo referenciado, el link simbólico sigue existiendo, 
haciendo referencia a un archivo inexistente. 

\subsection*{Código de tipos de archivos}
Varios comandos del sistema GNU/Linux, como ser \texttt{ls} y \texttt{stat}, utilizan un código 
para identificar en su salida, el tipo de archivo mostrado. La siguiente tabla muestra los 
códigos utilizados:

\definecolor{tcA}{rgb}{1,0.92549,0.623529}
\begin{center}
\begin{tabular}{llll}\hline
% use packages: color,colortbl
\rowcolor{tcA}
Tipo de archivo & Símbolo & Creado por & Eliminado por\\\hline
Archivo regular & - & aplicaciones,touch,cp,etc & rm\\
Directorio & d & mkdir & rm -r, rmdir\\
Link simbólico & l & ln -s & rm \\
Dispositivo (caracter y bloque) & c/b & mknod & rm\\
UNIX Sockets & s & socket(2) & rm\\
Pipes & p & mknod & rm\\\hline
\end{tabular}
\end{center}

\fcolorbox{black}{grey}{
\parbox[t]{1.0\linewidth}{ \vspace*{0.4cm}
{\bf Ejemplo} \\
Observe el primer caracter de las salidas de \texttt{ls}, donde se distingue
el código del tipo de archivo listado. \\
\texttt{
\$ ls -l /etc/rc2.d/S16rc.local \\
lrwxrwxrwx 1 root root 18 ene 26 20:24 /etc/rc2.d/S16rc.local -\textgreater ../init.d/rc.local \\
\$ ls -l lunes.txt \\
-rw-r--r-- 1 lechnerm lechnerm 27 abr  5 15:21 lunes.txt \\
\$ ls -l /dev/sda\\
brw-rw---- 1 root disk 8, 0 abr  4 10:34 /dev/sda \\
\$ ls -l /dev/tty\\
crw-rw-rw- 1 root root 5, 0 abr  5 16:16 /dev/tty \\
\$ ls -l /var/run/dbus/system\_bus\_socket \\
srwxrwxrwx 1 root root 0 abr  4 10:37 /var/run/dbus/system\_bus\_socket\\
\$ 
}
\vspace*{0.4cm} } }

\section*{Atributos}
El sistema de archivos mantiene distinto tipo de información acerca de cada archivo en él. 
Como administradores de sistemas, en este curso en particular, nos veremos interesados
por el tipo de archivo, la cantidad de enlaces a un archivo (ambos vistos en secciones 
anteriores), el propietario, el grupo al que pertenece y los permisos. 

Para observar estos atributos utilizamos el comando \texttt{ls -l} o \texttt{stat}. 
El comando \texttt{ls -l} muestra los permisos del archivo,  el  número  de  enlaces  
que  tiene,  el  nombre  del  propietario,  el  del  grupo  al  que  pertenece, 
el tamaño (en bytes), una marca de tiempo, y el nombre del archivo (véase man ls).


\fcolorbox{black}{grey}{
\parbox[t]{1.0\linewidth}{ \vspace*{0.4cm}
{\bf Ejemplo} \\
\texttt{
\$ ls -la .  \\
total 36 \\
drwxr-xr-x  3 lechnerm lechnerm  4096 abr  5 20:54 . \\
drwxr-xr-x 89 lechnerm lechnerm 16384 abr  5 19:34 .. \\
-rw-r--r--  1 lechnerm lechnerm    27 abr  5 15:21 lunes.txt \\
-rw-r--r--  1 root     root         0 abr  5 20:53 monitor.conf \\
-rw-r--r--  1 rafa     sudo       833 abr  5 20:52 practico.pdf \\
-rwxr-x---  1 lechnerm lechnerm     5 abr  5 20:54 script.sh \\
drwxr-xr-x  2 lechnerm lechnerm  4096 abr  5 18:56 test3 \\
\$ \\
}
Nota: la opción -a del comando \texttt{ls} muestra todos los archivos 
incluidos los ocultos, que son aquellos que comienzan con ``.''.  
\vspace*{0.4cm} } }

\subsection*{Propietario y grupo}
La tercera y cuarta columna de la salida de \texttt{ls -l} representa 
el usuario y el grupo al que el archivo 
pertenece respectivamente. En el ejemplo anterior el archivo 
practico.pdf pertenece al usuario rafa y al grupo sudo. Mientras que 
el archivo monitor.conf pertenece al usuario root y grupo root. 

De manera predeterminada, al crearse un archivo, este tendrá como 
dueño al usuario que ejecuta la aplicación que da origen al archivo y 
como grupo, el grupo primario al que pertenece el mismo usuario. 

En el ejemplo a continuación, el usuario lechnerm ejecuta el comando 
\texttt{cp} para crear una copia del archivo usars.py dentro del directorio
\texttt{/tmp}. El archivo original pertenece al usuario rafa, mientras que
la copia pertenece a lechnerm, ya que este fue el usuario que ejecuto la 
aplicación (comando) \texttt{cp} que dió origen a la copia: 

\fcolorbox{black}{grey}{
\parbox[t]{1.0\linewidth}{ \vspace*{0.4cm}
{\bf Ejemplo} \\
\texttt{
\$ id \\
uid=1000(lechnerm) gid=1000(lechnerm) grupos=1000(lechnerm),24(cdrom), \\
25(floppy),27(sudo),29(audio),30(dip),44(video),46(plugdev),\\
\$ ls -l /home/rafa/usars.py  \\
-rw-r--r-- 1 rafa rafa 205 oct  1  2013 /home/rafa/usars.py \\
\$ cp /home/rafa/usars.py /tmp/ \\
\$ ls -l /tmp/usars.py  \\
-rw-r--r-- 1 lechnerm lechnerm 205 abr  6 23:34 /tmp/usars.py \\
\$  
}
\vspace*{0.4cm} } }


Tanto el usuario como el grupo deben existir previamente. El comando 
que permite la modificación explícita de la propiedad de un archivo es \textbf{\texttt{chown}}
(del inglés ``\textbf{ch}ange \textbf{own}ership'').
Por cuestiones de seguridad, un usuario sin privilegios especiales (no root), 
no podrá cambiar el dueño de un archivo (aún cuando el archivo le pertenezca). 
Sólo podrá cambiar el grupo al que pertenece el archivo, siempre y cuando el usuario pertenezca a dicho grupo.  
Por otro lado, el administrador del sistema, root, podrá cambiar el dueño
y grupo sin restricciones. Siguiendo con el ejemplo anterior: 


\fcolorbox{black}{grey}{
\parbox[t]{1.0\linewidth}{ \vspace*{0.4cm}
{\bf Ejemplo} \\
\texttt{
\$ ls -l /tmp/usars.py  \\
-rw-r--r-- 1 lechnerm lechnerm 205 abr  6 23:34 /tmp/usars.py \\
\$ chown rafa.sudo /tmp/usars.py  \\
chown: cambiando el propietario de «/tmp/usars.py»: Operación no permitida \\
\$ chown lechnerm.sudo /tmp/usars.py  \\
\$ ls -l /tmp/usars.py  \\
-rw-r--r-- 1 lechnerm sudo 205 abr  6 23:34 /tmp/usars.py \\
\$ chown lechnerm.pulse /tmp/usars.py  \\
chown: cambiando el propietario de «/tmp/usars.py»: Operación no permitida \\
\$ sudo su - \\
\{ sudo \} password for lechnerm:  \\
\# chown rafa.rafa /tmp/usars.py  \\
\# ls -l /tmp/usars.py  \\
-rw-r--r-- 1 rafa rafa 205 abr  6 23:34 /tmp/usars.py \\
\# 
}
\vspace*{0.4cm} } }

\subsection*{Permisos}
Cada archivo tiene un conjunto de permisos asociados a él, los cuales 
determinan qué puede hacerse con el archivo y quién puede hacerlo. 

De manera simplificada podemos decir que los permisos de archivos 
están representados por tres números octales.
El primer número octal representa lo que puede hacer el dueño del 
archivo con el mismo; el segundo indica lo que puede hacer el grupo 
al que pertenece el archivo con el mismo; y el tercer conjunto indica 
lo que puede hacer cualquier otro usuario que 
no sea ni el dueño ni pertenezca al grupo (otros).

\fcolorbox{black}{grey}{
\parbox[t]{1.0\linewidth}{ \vspace*{0.4cm}
{\bf Ejemplo} \\
El archivo ``vcninja'' tiene permisos 644. Por ende el dueño, en este
caso lechnerm, tiene permiso 6, el grupo 4 y el resto del universo 4. \\
\texttt{
\$ stat vcninja \\
  Fichero: «vcninja» \\
  Tamaño: 24802     	Bloques: 56         Bloque E/S: 4096   fichero regular \\
Dispositivo: fe01h/65025d	Nodo-i: 5505447     Enlaces: 1 \\
\textbf{Acceso: (0644/-rw-r--r--)}  Uid: ( 1000/lechnerm)   Gid: ( 501/sudo) \\
      Acceso: 2013-10-16 14:40:00.076912329 -0300 \\
Modificación: 2013-10-01 16:02:30.002920012 -0300 \\
      Cambio: 2013-10-01 16:02:30.030919108 -0300 \\
    Creación: - \\
\$ ls -l vcninja \\
-rw-r--r-- 1 lechnerm sudo 24802 oct  1  2013 vcninja \\
\$ 
}
\vspace*{0.4cm} } }

La elección del sistema octal esta dada por los tres posibles permisos que 
un archivo puede tener. Estos permisos son:
\begin{itemize}
\item Lectura, simbolizado por la letra \textbf{r}
\item Escritura, simbolizado por la letra \textbf{w}
\item Ejecución, simbolizado por la letra \textbf{x}
\end{itemize}

Si asumimos que cada uno de estos tres permisos esta representado 
por un bit, tal que si el bit esta en 0 el permiso en cuestión es denegado
mientras que si esta en 1 el permiso es otorgado. Entonces necesitamos 
tres bits para representar las posibles combinaciones de esos tres permisos 
y por ende la elección de un número octal (dos al cubo es ocho). 

La siguiente tabla intenta aclarar lo explicado: 

\texttt{
\begin{center}
\begin{tabular}{lll}
\rowcolor{tcA}
Número octal & Representación binaria & Mapeo a letras\\
0 & 000  & ---\\
1 & 001 & --x\\
2 & 010 & -w-\\
3 & 011 & -wx\\
4 & 100 & r--\\
5 & 101 & r-x\\
6 & 110 & rw-\\
7 & 111 & rwx
\end{tabular}
\end{center}
}

Es así que en el ejemplo anterior: 

\texttt{
\$ ls -l vcninja \\
-rw-r--r-- 1 lechnerm sudo 24802 oct  1  2013 vcninja \\
}
Como vimos en la sección anterior, el primer bit representa el tipo de archivo, en este caso regular. Los 
siguientes tres bits representan los permisos para el dueño (lechnerm), en este 
caso \texttt{rw-}, equivalente al número octal 6. Los siguientes tres
bits representan los permisos para el grupo (sudo), en este caso {\tt r--}, 
equivalente al número 4 en octal. Los últimos tres bits son los 
permisos para otros, en este caso también {\tt r--}. 

\subsection*{Archivos regulares frente a directorios}

La interpretación de como funcionan los tres permisos, rwx, depende del 
tipo de archivo con el que estemos tratando. En particular nos interesa
lo que sucede con archivos regulares y directorios. 

El caso más sencillo es el de {\it archivos regulares}. Para ellos, la interpretación 
de {\tt rwx} es la intuitiva. Si {\tt r} se encuentra presente, entonces el contenido del archivo
puede ser leído (por ejemplo si es de texto a través del comando cat). Si {\tt w}
se encuentra presente entonces, el archivo en cuestión puede ser modificado 
(siguiendo con un texto, por ejemplo a través del editor vi). Si {\tt x} se encuentra
presente, entonces el archivo puede ser ejecutado. Este permiso deberá estar 
presente para archivos binarios ejecutables (por ejemplo los que están en /bin) y
cualquier otro archivo que pretendamos ejecutar, por ejemplo un script. 

En el ejemplo de la sección anterior, el archivo vcninja puede ser leído y 
modificado únicamente por lechnerm; y tanto el grupo como el resto de los 
usuarios del sistema sólo podrán leerlo. 

En el caso de {\it directorios}, si el permiso {\tt r} está presente
significa que el usuario puede leer el directorio, por lo que puede 
conocer qué archivos hay en él por medio del comando {\tt ls}. Si 
el permiso {\tt w} está presente significa que el usuario puede crear y 
borra archivos es este directorio, ya que para eso se requiere modificar y, 
por tanto, escribir sobe el archivo de tipo directorio.

En realidad, no es posible escribir simplemente en un archivo de tipo 
directorio (ni aún root puede hacerlo). Intentémoslo: 

\texttt{
\$ echo hola > .  \\
bash: .: Es un directorio \\
\$ mkdir test \\
\$ echo hola > test \\
bash: test: Es un directorio \\
\$  
}

Para modificar el contenido de un archivo de tipo directorio existen 
llamadas al sistema que crean y borran archivos, y sólo a través de 
ellas es posible cambiar el contenido de un directorio. Sin embargo, 
la idea de los permisos aún se aplica: cuando {\tt w} está presente 
es posible usar las aplicaciones del sistema para modificar el directorio, 
aplicaciones que terminarán invocando a las llamadas a sistema apropiadas. 

Los permisos para borrar archivos son independientes del archivo mismo. 
Si el usuario tiene permiso de escritura en un directorio, puede borrar 
archivos de él, aún los que estén protegidos contra escritura. Sin 
embargo, el comando rm pide al usuario confirmar antes de borrar
un archivo protegido para asegurar que eso es lo 
que en realidad desea hacer, siendo una de las raras ocasiones en las que 
un programa en UNIX verifica las intenciones del usuario. (La opción {\tt -f} de
{\tt rm } fuerza al comando a borrar archivos sin necesidad de confirmar.)

El permiso {\tt x} es quizá el menos intuitivo en cuanto a directorios. Este no 
significa ejecución; significa ``búsqueda''. El permiso de ejecución de 
un directorio determina si puede buscarse o no un archivo en él. Por lo tanto, 
es posible crear un directorio con modo {\tt --x} para otros usuarios, lo cual 
significa que puedan acceder cualquier archivo que conozcan 
previamente en ese directorio, pero no pueden ejecutar ls sobre él ni leerlo 
para ver qué archivos tiene. En forma similar, con permisos {\tt r--} los 
usuarios pueden ver ({\tt ls}) pero no usar el contenido de un directorio. 
Algunas instalaciones usan esto para desactivar /usr/games durante las horas de trabajo. 


\subsection*{Modificando permisos}
El comando {\tt chmod} (del inglés {\tt ch}ange {\tt mod}e), cambia los 
permisos de los archivos: 

{\tt chmod {\it permisos archivos..}}

Los permisos podrán especificarse de dos maneras, ya sea como números octales
(según la tabla vista previamente) o por descripción simbólica. 

\fcolorbox{black}{grey}{
\parbox[t]{1.0\linewidth}{ \vspace*{0.4cm}
{\bf Ejemplo} \\
Sólo el propietario del archivo y root podrán modificar los permisos de un archivo. 
\texttt{
\$ ls -l usars.py  \\
-rw-r--r-- 1 rafa rafa 205 abr  6 23:34 usars.py \\
\$chmod 755 usars.py  \\
\$ls -l usars.py  \\
-rwxr-xr-x 1 rafa rafa 205 abr  6 23:34 usars.py \\
\$ \# Al grupo le quitamos el permiso de ejecución \\
\$chmod g-x usars.py  \\
\$ls -l usars.py  \\
-rwxr--r-x 1 rafa rafa 205 abr  6 23:34 usars.py \\
\$ \# Al otros le quitamos el permiso de ejecución \\
\$chmod o-x  usars.py  \\
\$ls -l usars.py  \\
-rwxr--r-- 1 rafa rafa 205 abr  6 23:34 usars.py \\
\$ \# Agregamos permiso de ejecución para todos (all) \\
\$chmod a+x usars.py  \\
\$ls -l usars.py  \\
-rwxr-xr-x 1 rafa rafa 205 abr  6 23:34 usars.py \\
\$chmod 644 usars.py  \\
\$ls -l usars.py  \\
-rw-r--r-- 1 rafa rafa 205 abr  6 23:34 usars.py \\
\$ 
}
\vspace*{0.4cm} } }

Cada orden de cambio de modo  simbólico  empieza
       con  cero  o  más  letras  del  conjunto `ugoa': éstas controlan a qué
       usuarios se referirán los nuevos permisos del fichero  cuyos  permisos
       se van a cambiar: el usuario propietario (u), otros usuarios distintos
       del propietario pero del mismo grupo que el  del  fichero  (g),  otros
       usuarios  que ni son el propietario ni pertenecen al grupo del fichero
       (o), o todos los usuarios (a). De forma que `a' equivale aquí a `ugo'.


 El operador `+' hace que los permisos seleccionados se añadan a los ya
       existentes en cada fichero; `-' hace que  se  quiten  de  los  que  ya
       había; y `=' hace que sean los únicos que el fichero va a tener
(para más información {\tt man chmod}).



\section*{Licencia}
Esta obra está licenciada bajo una Licencia Creative Commons Atribución-CompartirDerivadasIgual 3.0 Unported. 

http://creativecommons.org/licenses/by-sa/3.0/


\end{document}

